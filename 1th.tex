\documentclass[runningheads]{llncs}
\usepackage{graphicx}
\usepackage{comment}
\usepackage[colorinlistoftodos]{todonotes}
\begin{document}
\title{Blockchain in the Industrial Sector}
\institute{RWTH Aachen University, Germany}
\author{Mohammad Ghanem}
\maketitle

\let\labelitemi\labelitemii

\section{Introduction \cite{Rojko2017}}

\paragraph{}

Industry 4.0 is a strategic initiative introduced by the German government to utilize new advanced technologies in the process of transforming traditional manufacturing into a digital, smart, and automated manufacturing.  The aim is to reduce production costs, increase efficiency and productivity, improve the quality of products, and achieve efficient use of natural resources and energy. To achieve all that a wide range of technologies has to be used in manufacturing processes like Robotics, Autonomous Systems, Internet of Things, Cloud Computing, Intelligent Data Analytics, Artificial Intelligence and many more. Integrating such technologies in factories comes with a lot of challenges as the manufacturing process involves different parties that operate in different countries, use different machines and facilities and have different interests. The data exchanged between such parties has to be secured, traceable, reliable, and trusted in order to establish effective and efficient business and manufacturing processes. Blockchain technologies can enable such exchange of data by providing a distributed ledger technology that can be used by different parties in industry 4.0 processes and procedures. Using blockchain in this context considered to be an evaluation toward having Industry 4.1 \cite{Internet2019}. Supply chain management and asset management systems are two examples where the blockchain ecosystem can solve real business problems.

In the second section, we will briefly show the features of using blockchain. The third section will focus on the benefits of implementing blockchain in some industry domains. 

\section{Blockchain \cite{Al-Jaroodi2019,Zheng2017}}
Blockchain technology has been evolving rapidly since 2008 when it has been introduced to be the underlying technology behind cryptocurrencies. Nowadays, blockchain is applicable and useful for many industrial use cases. It can be seen as an immutable, decentralized, distributed, digital ledger for storing transactions of any kind. There are many approaches to implement and build the blockchain and that involves a lot of technical and technological details. Regardless of that, blockchain, in general, has the following features:

\begin{itemize}
    \item \textbf{Transparency}: Data written to the blockchain in from of transactions are visible to all network parties.
    \item \textbf{Security \& Trust}: Using consensus and immutability algorithms makes the blockchain a 'trustless' system. 
    \item \textbf{Tractability}: Once the data has been written into the blockchain, it is extremely difficult to remove or change and that makes the blockchain perfect for storing data where an audit trail is required because every change is tracked and permanently recorded on a distributed and public ledger.
    \item \textbf{Decentralization}.
    \item \textbf{Distributed}.
\end{itemize}

\paragraph{}
Having all these features besides other technical details not mentioned here, blockchain can be used to provide different functionality depending on the data being written to it:

\begin{itemize}
\item \textbf{Static Registry}: Storing reference data in the blockchain like property registrations and patent intellectual property. This will show a complete history of ownership.
\item \textbf{Digital Identities}: Writing identity-related information to the blockchain will give the individuals and the organizations the ability to have a digital identity without the need for any identity mediators.
\item \textbf{Dynamic Registry}:  For storing dynamically generated data like financial transactions or data generated from the asset management systems.
\item \textbf{Smart Contracts}: Set of conditions in form of If/Then statements written to the blockchain. These conditions are self-executed when the predefined conditions are met. This permit conducting a credible any kind of contract without a third party.
\item \textbf{Payment}: Using cryptocurrency payments.
\end{itemize}

\section{Blockchain in Industry 4.0 \cite{Mohamed2019,Internet2019}}

\subsection{Supply Chain Management}
The core operation of the supply chain management systems depends on information, materials and financial flows between different processes and different stockholders. Typically this involves equipment manufacturers, suppliers, retailers, third-party logistics providers, shippers, and warehouses. The trust and transparency inside the supply chain flow are important to all stakeholders. All of them are interested in knowing the status of each process at any time e.g. the shipments location, storage, and movement of raw materials, product lifecycle. 

\paragraph{}

Using the blockchain to store the data about the goods, raw materials, information generated from sensors and tracking devices, financial transactions, business agreements can help to build a digitalized supply chain network. In this case, no more intermediaries in the supply chain as the blockchain is the source of truth and trust. The transactions invoked from IoT devices with the location and timestamps would serve as proof of shipment and proof of delivery. Origin and authenticity of raw materials and goods will be easily accessible as there is a full provenance history written inside the blockchain. Using smart contract to record any business contract will reduce administrative costs and provides more efficient ways to initiate, negotiate and finalize business deals without the need to rely on third-party or paperwork.



Until today, the industry has seen three major revolutions, and the fourth is on its way. The fourth industrial revolution represents the next step in the evolution of traditional factories towards smart automated factories. These factories are designed in such a way to reduce production costs, increase productivity, improve quality, and achieve efficient use of resources. One of the foundations of Industry 4.0 is gathering and sharing as much data as possible from the diverse parts of the value chain. This requires high levels of connectivity and communication to be established between the involved parties [Book 1].  In order to establish effective and efficient business and manufacturing processes, the exchanged information has to be secure, traceable, reliable, and trusted. Nothing of these can be achieved if the involved parties don’t trust each other. With the current digital technologies, the economy is based on the trust that already exists between the different parties or through trusted middlemen. This is not enough to unlock the full potential of Industry 4.0. A common and trusted platform is needed to facilitate the relationships between parties across the value chain. Such a platform could be achieved by using blockchain technology.

Initially, the primary applications of the blockchain were financial applications using cryptocurrencies. With the introduction of smart contracts, the applications of blockchain have no boundaries anymore. It can be used for both financial and nonfinancial applications [BlockChain Technology Beyond Bitcoin]. Given its key features such as immutability, traceability, and reliability of the information, it represents a perfect candidate to be integrated into the industrial plants. It will increase its efficiency, security, and provenance concerning the related data of goods, assets, and operations [Book 2]. Blockchain can also remove middle parties, secure communication, and build a digital reputation for a trustless secured business. Overall this technology has many things to offer depending on the underlying technical and technological details. 


As a result of Industry 4.0, In the manufacturing industry, many technologies are evolving. The major technologies are the Internet of Things (IoT), Cyber-Physical System, and Machine-to-machine (M2M) communication. 


They all have the same goal which is to increase the efficiency, capability, and adaptability of the manufacturing systems. Applying these technologies inside factories will turn them into smart factories. Where the production and manufacturing processes are automated and digitalized. They can be used in different industrial tasks such as automation, diagnostics, and management of industrial machines. M2M Communications is a concept which describes the technologies that allow networked machines to exchange information and execute actions without any human intervention—leading to full industry digitization. In an M2M enabled factory, a machine of one working level can communicate with a machine of another working-level via wireless or wired media without any human involvement.  For example, a printer can automatically order a color-supplying ink cartridge supplier machine when it runs out of ink. However, all technologies rely on a centralized network provided by a third party. This limits the ability of these technologies to achieve their goal and make the manufacturing systems not scalable. Also, this imposes vulnerabilities as cyber-attacks and security problems.  


 Such a system can be one step into building a blockchain-based ecosystem that can be used alongside the whole supply chain.

\subsection{Assets Management}

\bibliographystyle{splncs04}
\bibliography{ref.bib}

\end{document}